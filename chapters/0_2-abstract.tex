\chapter*{Abstract}

\chapter*{Abstract}

The Push-Pull Sum algorithm, introduced in \cite{nugroho2023PushPullSumDataAg}, combines elements of the Push-Sum \cite{kempe2003gossipbasedComp} and Pull-Sum algorithms. The Push-Sum algorithm, proposed by Kempe et al., is a load balancing method where each node randomly selects a neighbor and transfers half of its current sum and weight to that neighbor. Load balancing algorithms are designed to evenly distribute loads across networks, typically modeled as undirected graphs. In such networks, nodes exchange loads with their neighbors to achieve a balanced state. The Single-Proposal Deal-Agreement-Based load balancing algorithm, as presented in \cite{Dinitz2023DAB}, incorporates a deal-agreement mechanism into load transfers, ensuring fair exchanges between two nodes.

In this thesis, I propose and implement a variation of the Push-Pull Sum algorithm that integrates principles from the Deal-Agreement-Based algorithm and adaptive thresholding. This adaptation modifies and extends certain properties of the original Push-Pull Sum algorithm. For the Adaptive Threshold Push-Pull Sum algorithm, I provide pseudocode, implement the method in a peer-to-peer simulation environment, and analyze simulation outcomes across different topologies. The goal is to identify a compromise solution that achieves consistently strong performance across varying network structures. The performance of this algorithm is evaluated using the mean squared error (MSE) reduction over time as a convergence metric. Results are presented using log-log and log-linear graphs to compare the efficiency of error reduction in various scenarios. The slopes of the MSE curves offer insights into how effectively the algorithms distribute loads across the network. Additionally, the data is fitted to various models to assess the rate of convergence.

The findings demonstrate that the proposed modifications to the Push-Pull Sum algorithm yield a more efficient and scalable load balancing strategy for most of the scenarios tested in the experiments.
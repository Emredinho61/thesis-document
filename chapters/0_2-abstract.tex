\chapter*{Abstract}

The Push-Pull Sum algorithm, as introduced in \cite{nugroho2023PushPullSumDataAg}, combines the Push-Sum \cite{kempe2003gossipbasedComp} and Pull-Sum algorithms. The Push-Sum algorithm, proposed by Kempe et al., is a load balancing algorithm where each node randomly selects a neighbor and transfers half of its current sum and weight to that neighbor. The Push-Pull Sum algorithm is a randomized load balancing algorithm. Load balancing algorithms are used to average loads in networks, modeled as undirected graphs. In these networks, nodes exchange loads with their neighbors in order to reach a balanced network state. The Single-Proposal Deal-Agreement-Based load balancing algorithm as proposed in \cite{Dinitz2023DAB}, incorporates a Deal-Agreement into the load transfer to achieve only fair load transfers between two nodes.

In this thesis, I introduce and implement a variation of the Push-Pull Sum algorithm using principles of the Deal-Agreement-Based protocol and adaptive thresholding, designed to add or modify certain properties of the original Push-Pull Sum protocol. For the newly introduced load balancing approach, I provide the pseudocode, implement the algorithm in a peer-to-peer simulation tool, and analyze the simulation outcomes for different topologies. The objective is to find a compromise solution including overall good performance in different topologies.

The performance of this variations is evaluated using the mean squared error (MSE) reduction over time as a metric for convergence to the ground truth. The results are presented through log-log and log-linear graphs, which allow a comparison of the convergence rates and stability in different scenarios. The slope of the MSE curves reflects how efficiently the protocols distribute the load across the network. The data is fitted into different models to get an insight into the rate of convergence.

The findings suggest that modification of the Push-Pull Sum algorithm incorporates a more efficient and more scalable load balancing strategy for most of the conducted experiments.
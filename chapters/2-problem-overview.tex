\chapter{Problem Overview}\label{chap:problemoverview}
We consider an undirected general graph $G_r = (V, E_r \subseteq V \times V)$, where $V$ denotes the set of vertices (nodes) and $E_t$ denotes the set of edges at each round $r$. A load transfer between two nodes $u$ and $v$ may happen, if the two nodes are connected by an edge.

\section{Setting}\label{sec:setting}
The Peer-to-peer network is modeled as a static general graph, meaning that the set of edges may not change during the application of the load balancing algorithms. The main objective is to balance the state of the network. This objective will be achieved with local algorithms, so that each node only collects information of nodes in their direct neighborhood. While the Deal-Agreement-Based approach is solving the load balancing problem by determinism, the Push-Pull Sum protocol and the Adaptive Threshold Push-Pull Sum protocol use elements of randomness, to spread loads more evenly. The nodes for the Push-Pull Sum protocol are provided with the initial sum $s_{i,0}$ and weight $w_{i,0}$ values and list of neighbors as initial information. The Deal-Agreement-Based protocols nodes are provided with initial node values and a list of neighbors. The initial weight values for each node i $w_{i,0} = 1$. The sum of all weights $\sum_{i}{w_i,r}$ at any round r is equal to the network size $n$ and the sum $s_{i,0}$ is equal to the data input \cite{nugroho2023PushPullSumDataAg}. The setting is a contiunous setting, where data transfer over the edges may contain any amount not just integers. We assume an synchronous message delivery, where the time of message delivery is constant e.g. $O(1)$ time. Eventhough it is possible to make the asynchronous model synchronous by simply enlarging the time unit, we stick with the synchronous model, where each round takes $O(1)$ time.

\section{Approach}\label{sec:approach}
This research contains of three steps, the design of a load balancing algorithm composed of elements of two established load balancing algorithms, the simulation step to test the ability to balance the state of the network for distinct topologies, and a comparative analysis step, where the simulation outcomes are evaluated, using methods from statistics. In the previous research "Comparative Analysis of Load Balancing Algorithms in General Graphs" \cite{Bayazitoglu}, the strengths and weaknesses of two distinct load balancing algorithms were determined, by simulating them in different topologies for different network sizes to test the scalability and adaptability of the algorithms to different situations. The design of a novel adaptive threshold load balancing algorithm is based on this knowledge. Simulation outcomes for 30 distinct experiments for further statistical significance, since we operate with randomized algorithms, are gathered and finally evaluated using model fitting to determine the trend of the algorithms regarding mean squared error reduction, and slopes are calculated per round to see the consistency in mean squared error reduction. The results are presented in comprehensive plots, with ellaboration on why the plots look like they do.

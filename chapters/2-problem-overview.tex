\chapter{Problem Overview}\label{chap:problemoverview}
The load balancing problem is defined on an undirected general graph, where each node can transfer loads to its neighboring nodes via edges to achieve a balanced network state. The problem setting and the approach to address the problem are elaborated on in this section.

\section{Setting}\label{sec:setting}
\textbf{Continuous and Discrete}: In the continuous setting, nodes can transfer any amount of load over the edges, while in the discrete setting, all load transfers must consist of integer values.

\textbf{Synchronous and Asynchronous}:  In the synchronous setting, the time for message delivery is constant (e.g., O(1)), whereas in the asynchronous setting, the message delivery time can be unpredictably large. However, it is possible to convert an asynchronous setting into a synchronous one by adjusting the time frame within which messages are expected to be delivered.

\textbf{Static or Dynamic}: Load balancing algorithms can operate in either static or dynamic graph settings. In a dynamic graph, connections between nodes may change arbitrarily between rounds. In contrast, the connections and the nodes remain the same for the static graph.

The peer-to-peer network is modeled as a static general graph, meaning the set of edges remains unchanged during the application of the load balancing algorithms. The objective of load balancing is achieved using local algorithms, where each node gathers information only from its direct neighbors. The setting is a continuous setting. Additionally, a synchronous message delivery assumption is made, with a constant delivery time, e.g., O(1). The experiments are conducted on six distinct network topologies, each comprising $2^{10}$ nodes.

\section{Approach}\label{sec:approach}
This research consists of three steps: the design of a load balancing algorithm, the simulation phase to test its ability to balance the state of the network across different topologies, and a comparative analysis of the simulation outcomes using statistical methods. In prior research \cite{Bayazitoglu}, the strengths and weaknesses of two distinct load balancing algorithms were identified by simulating their performance on various topologies and network sizes to test the scalability and adaptability of the algorithms to different situations. The design of the novel adaptive threshold load balancing algorithm builds upon these findings. Each simulation outcome includes 30 distinct experiments to ensure statistical significance. The results are analyzed using the concept of model fitting to identify trends in MSE reduction, with slopes calculated for different regions to assess the consistency of performance. The findings are presented in plots, accompanied by explanations that provide insights into the observed behavior of the load balancing algorithms.

\chapter{Problem Overview}\label{chap:problemoverview}
The load balancing problem is defined in an undirected general graph, where each load may transfer loads over the edges to their neighboring nodes in order to balance the state of the network. The setting and the approach are elaborated on in this section.

\section{Setting}\label{sec:setting}
\begin{itemize}
    \item \textbf{Continuous and Discrete}: In the Continuous setting, load balancing algorithms may transfer any amount of load over the edges, while in the discrete setting all load transfers should contain integers.
    \item \textbf{Synchronous and Asynchronous}: In the synchronous setting the time of message delivery is constant (e.g. O(1)), while in the asynchronous setting the time of message delivery may be unpredictably large. However, it is possible to make the asynchronous setting synchronous, by simply adjusting the time frame in which messages should be delivered.
    \item \textbf{Static or Dynamic}: The graphs in which the load balancing algorithms operate in, may be static or dynamic. A dynamic graph setting may change arbitrarily between the rounds, while the static connections and the nodes remain the same for the static graph.
\end{itemize}

The Peer-to-peer network is modeled as a static general graph, meaning that the set of edges may not change during the application of the load balancing algorithms. The objective of balancing load will be achieved with local algorithms, so that each node only collects information of nodes in their direct neighborhood. The setting is a continuous setting, where data transfer over the edges may contain any amount not just integers. The assumption of an synchronous message delivery is made, where the time of message delivery is constant e.g. $O(1)$ time and thus is in a synchronous setting. There are six distinct network topologies under test, each consists of $2^{10}$ nodes.

\section{Approach}\label{sec:approach}
This research consists of three steps, the design of a load balancing algorithm, the simulation step to test the ability to balance the state of the network for distinct topologies, and a comparative analysis step, where the simulation outcomes are evaluated, using methods from statistics. In the previous research "Comparative Analysis of Load Balancing Algorithms in General Graphs" \cite{Bayazitoglu}, the strengths and weaknesses of two distinct load balancing algorithms were determined, by simulating them in different topologies for different network sizes to test the scalability and adaptability of the algorithms to different situations. The design of a novel adaptive threshold load balancing algorithm is based on this knowledge. Each simulation contains of 30 distinct experiments, for further statistical significance. The simulation outcomes are finally evaluated using model fitting to determine the trend of the algorithms regarding MSE reduction, and slopes are calculated per region to see the consistency in MSE reduction. The results are presented in comprehensive plots, with ellaboration on why the plots look like they do.

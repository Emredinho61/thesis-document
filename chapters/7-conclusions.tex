\chapter{Conclusion}\label{chap:conclusion}
The Adaptive Threshold Push-Pull Sum algorithm shows to be a compromise solution between the Push-Pull Sum and Single-Proposal Deal-Agreement-Based algorithms. The simulation results show that the Adaptive Threshold Push-Pull Sum algorithm performs well in dense graphs as well as in regular low-degree graphs. Especially for the Complete Graph $K_{1024}$, Ring of Cliques $ROC_{32,32}$ and $ROC_{8,128}$, as well as for Lollipop $L_{512,512}$ and $L_{896,128}$ topologies, the ATPPS algorithm proved to be the best solution, as the algorithm achieved the lowest MSE values after 100 rounds of simulation. The Adaptive Threshold Push-Pull Sum achieved good overall results. In cases where the PPS failed to show good balancing abilities, like in the $ROC_{32,32}$, the Adaptive Threshold Push-Pull Sum proved to be very efficient, achieving results close to the Deal-Agreement-Based algorithm. In cases where the Push-Pull Sum algorithm achieved lower MSE within 100 rounds, the difference to the Adaptive Threshold Push-Pull Sum algorithm was not significant as seen for the Star graph $S_{1024}$, the Lollipop graph $L_{128,896}$, as well as the Torus Grid $T_{32,32}$. However, when the Adaptive Threshold Push-Pull Sum outperforms the traditional Push-Pull Sum algorithm, the MSE discrepancies are more pronounced, like in the case of Ring of Cliques $ROC_{32,32}$ and $ROC_{128,8}$. Here, the Adaptive Threshold Push-Pull Sum algorithm manages to distribute loads efficiently by adapting to the network state. When the load differences within the cliques were no longer significant enough, clique-to-clique communication via the bridging nodes was favored. The Adaptive Threshold Push-Pull Sum algorithm also achieves the sharpest downward trend for $K_{1024}$ until the curve finally stagnates (due to the precision of doubles in Java). No improvement of the load balancing behavior was observed for the Ring structure $R_{1024}$. The reason for this is that each node $i$ has exactly 2 neighbors, and thus only one neighbor is selected as $\lceil\log_{2}{(|neighborhood_{i}|)}\rceil$ evaluates to 1. The behavior of this algorithm is therefore very similar to the traditional Push-Pull Sum algorithm, without prioritizing any nodes. Also for the Star topology, the Adaptive Threshold Push-Pull Sum algorithm does not achieve any advantage over the traditional Push-Pull Sum algorithm, as the leaf nodes all communicate with the central node. The condition with significant load transfers limits the number of requests to the central node compared to the traditional Push-Pull Sum algorithm. The result is that the Push-Pull Sum algorithm achieves a balanced state of the network earlier. Compared to the Push-Pull Sum algorithms, the Single-Proposal Deal-Agreement-Based algorithm has problems with dense graphs, such as the Complete graph, Lollipop graph with a large clique size, and Ring of Cliques with a large clique size. The MSE differences for dense graphs after 100 rounds of computation are extremely large between the Single-Proposal Deal-Agreement-Based algorithm and the Push-Pull Sum-based algorithms. For low-degree topologies such as the Ring graph and the Torus Grid graph the MSE differences are not as significant, especially between the best-performing Single-Proposal Deal-Agreement-Based algorithm and the Adaptive Threshold Push-Pull Sum algorithm. The simulation results depend mainly on the sensitivity factor $k$. A large $k$ ensures a more sensitive selection of neighbors, while a smaller $k$ allows a coarser selection of neighbors.
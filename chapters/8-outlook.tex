\chapter{Outlook}\label{chap:outlook}
The proposed load balancing algorithm can be used in many different scenarios, like in cloud environments where many different servers are connected to a network, many with different hardware specifications. The method of randomized neighbor selection and the push and pull mechanisms allow an even distribution of loads throughout the network. The adaptive threshold condition only allows load transfers with a high impact on the balance of the network. The weight value can, for instance, represent the server capacities, i.e., given the hardware of a server, how capable it is of balancing the loads relative to the other partners in the network.

It is important to note that the performance of the algorithm was only tested in specific static topologies. Future research can also inspect dynamic networks. Furthermore, the performance was only tested with the metric of mean squared error, and the trend of its decay was analyzed with model fitting. However, further research is planned to compare the number of messages sent to achieve low imbalance in the network after r rounds. This would provide an insight into how efficient the load balancing mechanisms of the individual algorithms are.